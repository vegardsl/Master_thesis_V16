\chapter{Discussion}
\label{chp:discussion} 

\section{Mapping}

\begin{itemize}
\item Repairing broken maps? What to do when map is partially broken.
\item Poor odometry when surrounding are in motion, or when laser features are difficult to detect. System can be fooled easily. 
\end{itemize}

\section{Navigation}
\begin{itemize}
\item Rectangle base vs. square base.
\item holonomic wheel.
\item Open loop wheel control (stuck when friction is high.)
\item Same as for mapping. Poor odometry when surrounding are in motion, or when laser features are difficult to detect. System can be fooled easily. 
\end{itemize}


\section{Suitability for Offshore Maintenance}

This is just a prototype. Mobility issues.
Kinect-like sensors and ROS could be useful. It is at least an excellent tool for ''rapid'' prototyping.

\subsection{Open Source Software and Security}

\ac{ROS} and other open source projects thrive on active communities of contributors. Both \ac{PCL} and \ac{ROS}, as well as many other libraries and frameworks, are built on a collaborative effort from researchers and developers across the globe. This is open structure is great for speeding up innovation. Issues and bugs can also be discovered more quickly by anyone. Another benefit is that every detail in an open source project is open for scrutiny by those who want to use it. This is also a problem in terms of security. While anyone can find bugs and issues, the code is also open to those who are looking for possible exploits and vulnerabilities. If a system is targeted for sabotage, and it is widely known that the system uses open source software, it might be more vulnerable to security threats.

