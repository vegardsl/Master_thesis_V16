\chapter{Setting Up the Project}

\section{Installation}

\subsection{Equipment List}

\subsubsection{Item List}

\subsubsection{Software list}

\begin{itemize}
	\item[Hector SLAM for ROS] Install with sudo apt-get install ros-indigo-hector-slam
\end{itemize}

\subsubsection{Compatibility Issues}

Indigo, Ubuntu etc.

\subsection{Install Ubuntu}

\subsection{Download ROS}

\section{Configuring the Project}

\subsection{Configuring the ROS Workspace}

\subsection{Configuring the Bluetooth Connection}

The Qt framework is used to simplify the implementation of the Bluetooth connection between the \ac{ROS} graph and a remote device. Our \ac{ROS} installation for this project already includes some variant of Qt version 4.8. While useful for creating new \ac{GUI} applications, it lacks a Bluetooth API. The latest version of Qt, version 5.5, is equipped with libraries necessary for developing Bluetooth applications. This part of the guide explain how to create a Qt 5 application which can be build by \textit{catkin\_make} and run as a \textit{rosnode}.

\subsection{Adding the Qwt Plugin For Customizing QDial}

This is a ''howto'' guide on how to install the Qwt-plugin.The following is assumed:

\begin{itemize}
	\item Development is performed on Windows 7
	\item Qt 5.x is installed at c:/Qt
\end{itemize}